\documentclass[a4paper, 10pt]{article}


% Typography

%% Enable typography stuff
\usepackage[no-math]{fontspec}
\defaultfontfeatures{Mapping = tex-text}


% Polyglossia

\usepackage{polyglossia}
\setdefaultlanguage{english}
\setotherlanguage{russian}


% Fonts

\setromanfont{Linux Libertine O}

%% Headings

\usepackage{sectsty}
\usepackage[normalem]{ulem}

\sectionfont{\mdseries\upshape\Large}
\subsectionfont{\scshape\normalsize}
\subsubsectionfont{\mdseries\upshape\large}


% Page setup

%% Pretty page setup
\usepackage{geometry}

%% Remove indentation
\setlength\parindent{0em}

%% Set line spacing
\usepackage{setspace}
\setstretch{1.1}


% Colors

\usepackage{xcolor}

%% Tango color palette

\definecolor{DarkSkyBlue}{HTML}{204A87}


% PDF configuration

%% * breaklinks — allow links to break across lines
%% * xetex — backend
%% * bookmarks — provide PDF bookmarks
\usepackage[breaklinks, xetex, bookmarks]{hyperref}

%% * linkcolor — local links
%% * citecolor — citations
%% * urlcolor — URLs
\hypersetup {
  colorlinks,
  linkcolor = DarkSkyBlue,
  citecolor = DarkSkyBlue,
  urlcolor = DarkSkyBlue,
}

%% Don’t use Mono font for URLs
\urlstyle{same}



\newcommand{\person}{Артур Дрёмов}
\newcommand{\subject}{CV}

\hypersetup {
  pdfauthor = {\person},
  pdfcreator = {XeTeX},
  pdfsubject = {\subject},
  pdftitle = {\person~--- \subject}
}


\begin{document}

  {\LARGE Артур Дрёмов} \\ [1em]
  Разработчик приложений для ОС Android \\ [1em]

  Новополоцк, Беларусь \\
  \href{mailto:artur.dryomov@gmail.com}{artur.dryomov@gmail.com} \\


  \section*{Навыки и умения}

    \subsection*{Языки программирования}

      \begin{itemize}

        \item Java~--- разработка приложений для ОС Android.

        \item Python, Ruby~--- небольшие проекты, скрипты, утилиты.

        \item Objective-C~--- проекты на Cocoa и правки на Cocoa Touch.

        \item C, C++~--- задания в университете (включая использование Qt).

        \item Zsh, Bash~--- утилиты системного администрирования.

        \item Haskell~--- изучался для расширения кругозора.

      \end{itemize}

    \subsection*{Языки разметки}

      \begin{itemize}

        \item \TeX\ ~--- документы, заметки, отчёты, \href{https://github.com/ming13/cv/}{это CV}.

        \item HTML, CSS, JavaScript~--- небольшие сайты и страницы.

      \end{itemize}

    \subsection*{Иные языки}

      \begin{itemize}

        \item SQL~--- PostgreSQL, MySQL и SQLite.

      \end{itemize}

    \subsection*{Вспомогательные средства}

      \begin{itemize}

        \item Git~--- использую каждый день для всех проектов, которыми занимаюсь.

        \item Make, CMake, Maven, Gradle~--- стараюсь не зависеть от IDE.

      \end{itemize}

    \subsection*{Операционные системы}

      \begin{itemize}

        \item Linux~--- 5 лет в качестве основной ОС в виде Arch Linux.

        \item OS X~--- текущая основная ОС, используется в~течение 2 лет.

      \end{itemize}


  \section*{Образование}

    Полоцкий государственный университет, Беларусь (2009~--- 2014) \\
    Программное обеспечение информационных технологий


  \section*{Языки}

    \begin{itemize}

      \item Английский~--- свободное чтение и письмо, разговорный выше среднего уровня.

      \item Русский~--- родной.

    \end{itemize}


  \section*{Профессиональный опыт}

      \textbf{Amahi}~--- доступно \href{https://play.google.com/store/apps/details?id=org.amahi.anywhere}{в Google Play}
        и \href{https://github.com/amahi/android}{на GitHub} \\

        Реализовано в рамках \href{https://developers.google.com/open-source/soc}{программы Google Summer of Code 2014}.
        Приложение-клиент для домашних серверов под управлением Amahi, являющееся
        просмотрщиком файлов на серверах. Доступ к данным осуществляется
        посредством RESTful API. Приложение имеет поддержку стриминга
        видео и аудио посредством библиотеки LibVLC и встроенных в ОС
        средств соответственно. Занимался дизайном, проектированием и реализацией. \\

      \textbf{LibreOffice Impress Remote}~--- доступно \href{https://play.google.com/store/apps/details?id=org.libreoffice.impressremote}{в Google Play} \\

        Реализовано в рамках \href{https://developers.google.com/open-source/soc}{программы Google Summer of Code 2013}.
        Мобильное приложение, позволяющее удалённо управлять презентациями
        LibreOffice при помощи WiFi или Bluetooth соединения. В рамках
        Google Summer of Code была проведёна полная реорганизиция приложения,
        включающая в себя обновлённый дизайн согласно последним требованиям
        Google, улучшение стабильности и скорости работы, намечены новые
        функции и версия для планшетов. \\

      \textbf{Bus Time}~--- доступно \href{https://play.google.com/store/apps/details?id=ru.ming13.bustime}{в Google Play} \\

        Расписание автопарка города Новополоцка в виде мобильного приложения.
        Первые версии не имели готовой базы расписаний, поэтому приложение
        предоставляло возможность редактирования информации с последующей синхронизацией
        с сервисом \href{https://dropbox.com}{Dropbox}. Текущая версия
        предоставляет готовые данные о рейсах. Занимался дизайном, разработкой
        и сбором информации, так как автопарк не имеет открытого API для доступа
        к необходимым данным. \\

      \textbf{Gambit}~--- доступно \href{https://play.google.com/store/apps/details?id=ru.ming13.gambit}{в Google Play}
        и \href{https://github.com/ming13/gambit}{на GitHub} \\

        Приложение, предназначенное для помощи изучения иностранных языков,
        используя словарные карточки. Приложение имеет возможность
        синхронизации с сервисом \href{https://drive.google.com}{Google Drive}.
        Занимался дизайном
        и разработкой. Данное приложение является открытым программным обеспечением
        и доступно \href{https://github.com/ming13/gambit}{на GitHub},
        служа в том числе примером моего подхода к организации исходного кода.

    \subsection*{Университет, открытое ПО и иные проекты}

      В рамках учебного процесса были разработаны несколько проектов,
      самыми значимыми являются
      информационная система расписания движения общественного транспорта,
      реализованная при помощи Ruby, Ruby on Rails, Grape и PostgreSQL,
      RESTful веб-сервер для организации доступа к данным аэропорта, реализованный
      при момощи Python, SQLAlchemy, PostgreSQL и Flask, а также
      \href{https://github.com/ming13/aequatio}{математическая утилита},
      реализованная при помощи C++ и Qt. \\

      В качестве фриланс-разработчика
      мной были реализованы и другие проекты в виде мобильных приложений,
      например, клиент для хостинг-провайдера на основе имеющегося
      JSON API, каталог продукции экстерьерного дизайна и справочная система
      общественных заведений.
      Некоторые проекты были реализованы на языке программирования Python~--- примером может послужить
      асинхронный XML-RPC веб-сервер с использованием Twisted и Wokkel. \\

      В свободное время занимаюсь разработкой открытого ПО.
      В основном, это небольшие патчи, например, для проектов
      \href{https://github.com/square/fest-android/commits?author=ming13}{Android FEST},
      \href{https://github.com/jdamcd/musicbrainz-android/commits?author=ming13}{MusicBrainz},
      \href{https://gerrit.libreoffice.org/#/q/owner:%22Artur+Dryomov%22+status:closed,n,z}{LibreOffice Impress},
      однако, при наличии возможности, я стараюсь продвигать и значительные изменения, примерами
      которых могут послужить
      \href{https://github.com/karlseguin/the-little-mongodb-book/pull/16}{сборка The Little MongoDB Book} и
      \href{https://github.com/keyboardsurfer/Crouton/pulls/ming13?state=closed}{поддержка анимаций и Maven для Crouton}.


  \section*{О себе}

    \subsection*{Характеристика}

      Трудолюбив, стараюсь постоянно изучать новое.
      Мои интересы: разработка мобильных приложений, дизайн
      взаимодействия, UNIX-подобные операционные системы.

    \subsection*{Подробно}

      Разработка для ОС Android на данный момент является моей основной специализацией.
      Предпочитаю заниматься дизайном приложений <<с нуля>>, следуя
      \href{https://developer.android.com/design/}{официальным руководствам}.
      Помимо общепринятых рекоммендаций, применяю идеи Джефа Раскина
      и Алана Купера. Для разработки предпочитаю использовать стек
      открытых библиотек, основанный на Maven или Gradle (примером является
      \href{https://github.com/ming13/gambit}{Gambit}). \\

      Являюсь сторонником UNIX-подобных операционных систем,
      таких как Linux и OS~X. Стараюсь оптимизировать как рабочий процесс,
      так и разрабатываемые продукты. \\

  \vfill

\end{document}
