\documentclass[a4paper, 10pt]{article}


% Typography

%% Enable typography stuff
\usepackage[no-math]{fontspec}
\defaultfontfeatures{Mapping = tex-text}


% Polyglossia

\usepackage{polyglossia}
\setdefaultlanguage{english}
\setotherlanguage{russian}


% Fonts

\setromanfont{Linux Libertine O}

%% Headings

\usepackage{sectsty}
\usepackage[normalem]{ulem}

\sectionfont{\mdseries\upshape\Large}
\subsectionfont{\scshape\normalsize}
\subsubsectionfont{\mdseries\upshape\large}


% Page setup

%% Pretty page setup
\usepackage{geometry}

%% Remove indentation
\setlength\parindent{0em}

%% Set line spacing
\usepackage{setspace}
\setstretch{1.1}


% Colors

\usepackage{xcolor}

%% Tango color palette

\definecolor{DarkSkyBlue}{HTML}{204A87}


% PDF configuration

%% * breaklinks — allow links to break across lines
%% * xetex — backend
%% * bookmarks — provide PDF bookmarks
\usepackage[breaklinks, xetex, bookmarks]{hyperref}

%% * linkcolor — local links
%% * citecolor — citations
%% * urlcolor — URLs
\hypersetup {
  colorlinks,
  linkcolor = DarkSkyBlue,
  citecolor = DarkSkyBlue,
  urlcolor = DarkSkyBlue,
}

%% Don’t use Mono font for URLs
\urlstyle{same}



\newcommand{\person}{Artur Dryomov}
\newcommand{\subject}{CV}

\hypersetup {
  pdfauthor = {\person},
  pdfcreator = {XeTeX},
  pdfsubject = {\subject},
  pdftitle = {\person~--- \subject}
}


\begin{document}

  {\LARGE Artur Dryomov} \\ [1em]
  Android Developer \\ [1.5em]

  Novopolotsk, Belarus \\
  +375 29 294-17-90 \\
  \href{mailto:artur.dryomov@gmail.com}{artur.dryomov@gmail.com} \\


  \section*{Professional Experience}

    \subsection*{Personal Projects}

      \textbf{Bus Time}~--- available \href{https://play.google.com/store/apps/details?id=ru.ming13.bustime}{on Google Play} \\

        The timetable for the city where I live in as an Android application.
        The very first version hadn’t any ready-to-use information, so the app
        had the ability to edit timetables and sync them via
        \href{https://dropbox.com}{Dropbox}. The current version contains
        all information users need in a pretty form. I designed and developed
        it myself, plus collected information about buses, because Novopolotsk
        doesn’t have any centralized API for obtaining this. \\

      \textbf{Gambit}~--- availabe \href{https://play.google.com/store/apps/details?id=ru.ming13.gambit}{on Google Play}
        and \href{https://github.com/ming13/gambit}{GitHub} \\

        Android application for learning foreign languages
        (and other things if you want) using flashcards.
        The first version of the application had the ability of
        with \href{https://drive.google.com}{Google Drive} sync, the current version
        doesn’t support it because of API changes. I designed and developed
        it myself too. It is open source, so you could read all the code
        \href{https://github.com/ming13/gambit}{on GitHub} and understand
        how I prefer to organize Android application development in practice.

    \subsection*{University and Freelance Projects}

      The most noticeable course works at the university were
      a \href{https://github.com/ming13/aequatio}{mathematical tool} written
      using C++ and Qt and a RESTful web-server using Python,
      SQLAlchemy, PostgreSQL and Flask. \\

      As a freelancer I wrote some Android applications, the last one was
      a client for a hosting provider using their JSON API. Except Android
      development there were Python projects, for example, an XML-RPC
      web-server using Twisted. \\

      I can tell you more if you are interested, so do not hesitate to ask
      questions.

  \section*{Technical Skills}

    \subsection*{Programming Languages}

      \begin{itemize}

        \item Java~--- Android applications development.

        \item Python~--- small projects, scripts, utilities.

        \item C, C++~--- tasks in university (including Qt usage).

        \item Zsh, Bash~--- system administation utilities.

        \item Haskell, Objective-C~--- it was fun.

      \end{itemize}

    \subsection*{Markup Languages}

      \begin{itemize}

        \item \TeX\ ~--- documents, notes, reports, this CV.

        \item HTML/CSS~--- small sites for my needs.

      \end{itemize}

    \subsection*{Other Languages}

      \begin{itemize}

        \item SQL~--- PostgreSQL, MySQL and SQLite.

      \end{itemize}

    \subsection*{Tools}

      \begin{itemize}

        \item Git~--- for everyday usage for all code I write.

        \item Make, CMake, Maven~--- I don’t rely on IDEs.

      \end{itemize}

    \subsection*{Operating Systems}

      \begin{itemize}

        \item Linux~--- from 2007 till now, Arch Linux is my personal choice.

        \item OS X~--- periodical usage, I could switch to it with no harm.

        \item Windows~--- saw it on screenshots in the internet.

      \end{itemize}


  \section*{Education}

    Polotsk State University, Belarus (2009~--- \dots) \\
    Computer Science

  \section*{Languages}

    \begin{itemize}

      \item Russian~--- native.

      \item English~--- fluent in reading and writing, speaking is not my best.

    \end{itemize}

  \section*{Personal}

    \subsection*{Brief}

      Hard-working and explorative. My interests are:
      mobile application development, UI design, UNIX-like
      operating systems.

    \subsection*{In detail}

      Android development is my main specialization. I prefer designing
      applications from scratch on my own following
      \href{https://developer.android.com/design/}{Android design guidelines}
      because other designers prefer to move iOS design elements to Android
      and the result is a crap. Except guidelines I follow Jef Raskin
      and Alan Cooper ideas. For development I use open-source libraries
      stack based on Maven build system (look at
      \href{https://github.com/ming13/gambit}{Gambit’s source code} for example). \\

      I’m a big fan of Linux and I use it for many years as
      an only operating system, so any work with UNIX-like operating systems
      would be a great pleasure and productivity boost for me.
      I like automatization everywhere I could do it and I try to optimize
      work process and software too. Open source is fun for me and
      I don’t hesitate for small or large contributions. \\

      Every free time I enjoy learning new ideas, projects and technologies
      which correlate with my interests and ready to work with them.
      If you have some questions to me~--- I’m ready to answer.

  \vfill

\end{document}
