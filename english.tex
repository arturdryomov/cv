\documentclass[a4paper, 10pt]{article}


% Typography

%% Enable typography stuff
\usepackage[no-math]{fontspec}
\defaultfontfeatures{Mapping = tex-text}


% Polyglossia

\usepackage{polyglossia}
\setdefaultlanguage{english}
\setotherlanguage{russian}


% Fonts

\setromanfont{Linux Libertine O}

%% Headings

\usepackage{sectsty}
\usepackage[normalem]{ulem}

\sectionfont{\mdseries\upshape\Large}
\subsectionfont{\scshape\normalsize}
\subsubsectionfont{\mdseries\upshape\large}


% Page setup

%% Pretty page setup
\usepackage{geometry}

%% Remove indentation
\setlength\parindent{0em}

%% Set line spacing
\usepackage{setspace}
\setstretch{1.1}


% Colors

\usepackage{xcolor}

%% Tango color palette

\definecolor{DarkSkyBlue}{HTML}{204A87}


% PDF configuration

%% * breaklinks — allow links to break across lines
%% * xetex — backend
%% * bookmarks — provide PDF bookmarks
\usepackage[breaklinks, xetex, bookmarks]{hyperref}

%% * linkcolor — local links
%% * citecolor — citations
%% * urlcolor — URLs
\hypersetup {
  colorlinks,
  linkcolor = DarkSkyBlue,
  citecolor = DarkSkyBlue,
  urlcolor = DarkSkyBlue,
}

%% Don’t use Mono font for URLs
\urlstyle{same}



\newcommand{\person}{Artur Dryomov}
\newcommand{\subject}{CV}

\hypersetup {
  pdfauthor = {\person},
  pdfcreator = {XeTeX},
  pdfsubject = {\subject},
  pdftitle = {\person~--- \subject}
}


\begin{document}

  {\LARGE Artur Dryomov} \\ [1em]
  Android Developer \\ [1.5em]

  Novopolotsk, Belarus \\
  \href{mailto:artur.dryomov@gmail.com}{artur.dryomov@gmail.com} \\


  \section*{Technical Skills}

    \subsection*{Programming Languages}

      \begin{itemize}

        \item Java~--- Android applications development.

        \item Python~--- small projects, scripts, utilities.

        \item C, C++~--- tasks in university (including Qt usage).

        \item Zsh, Bash~--- system administration utilities.

        \item Haskell, Objective-C, Ruby~--- it was fun.

      \end{itemize}

    \subsection*{Markup Languages}

      \begin{itemize}

        \item \TeX\ ~--- documents, notes, reports, \href{https://github.com/ming13/cv/}{this CV}.

        \item HTML/CSS~--- small sites for my needs.

      \end{itemize}

    \subsection*{Other Languages}

      \begin{itemize}

        \item SQL~--- PostgreSQL, MySQL and SQLite.

      \end{itemize}

    \subsection*{Tools}

      \begin{itemize}

        \item Git~--- for everyday usage for all code I write.

        \item Make, CMake, Maven, Gradle~--- I do not rely on IDEs.

      \end{itemize}

    \subsection*{Operating Systems}

      \begin{itemize}

        \item Linux~--- from 2007 till 2013 as an only OS, Arch Linux is my personal choice.

        \item OS X~--- the current OS.

      \end{itemize}


  \section*{Education}

    Polotsk State University, Belarus (2009~--- \dots) \\
    Computer Science


  \section*{Languages}

    \begin{itemize}

      \item English~--- fluent in reading, writing and speaking.

      \item Russian~--- native.

    \end{itemize}


  \section*{Professional Experience}

    \subsection*{Community Projects}

      \textbf{LibreOffice Impress Remote}~--- available \href{https://play.google.com/store/apps/details?id=org.libreoffice.impressremote}{on Google Play} \\

        Implemented as a part of the \href{https://developers.google.com/open-source/soc}{Google Summer of Code 2013 program}.
        Android application as a presentation remote control for LibreOffice Impress.
        The app allows users to interact with their slideshows using Android device
        via WiFi or Bluetooth connection. During the program the application was
        rewritten for better performance and stability as well as more native
        UI matching the latest design guidelines. \\

    \subsection*{Personal Projects}

      \textbf{Bus Time}~--- available \href{https://play.google.com/store/apps/details?id=ru.ming13.bustime}{on Google Play} \\

        The timetable for the city I live in as an Android application.
        The very first version did not have any ready-to-use information so the app
        had the ability to edit timetables and sync them via
        \href{https://dropbox.com}{Dropbox}. The current version contains
        the information users need in a pretty form. I designed and developed
        it by myself, as well as collected information about buses because Novopolotsk
        does not have any centralized API for obtaining it. \\

      \textbf{Gambit}~--- available \href{https://play.google.com/store/apps/details?id=ru.ming13.gambit}{on Google Play}
        and \href{https://github.com/ming13/gambit}{GitHub} \\

        Android application for learning foreign languages
        (and other things if you wish) using flashcards.
        The app has the ability of \href{https://drive.google.com}{Google Drive} sync.
        I designed and developed
        it myself as well. It is an open-source software, so you could read the code
        \href{https://github.com/ming13/gambit}{on GitHub} and understand
        how I prefer to organize Android application development in practice.

    \subsection*{University, Freelance and Open Source}

      The most noticeable course works at the university were the writing of
      \href{https://github.com/ming13/aequatio}{a mathematical tool}
      using C++ and Qt and a RESTful web-server using Python,
      SQLAlchemy, PostgreSQL and Flask. \\

      As a freelancer I wrote some Android applications, the last one was
      a client for a hosting provider using their JSON API
      and Google HTTP Java client.
      Except Android development there were Python projects, for example,
      an async XML-RPC web-server using Twisted and Wokkel. \\

      Open source is fun for me, I like GitHub and I am not shy to prepare
      small or large patches. Basically they are just refactoring like
      \href{https://github.com/square/fest-android/commits?author=ming13}{Android FEST},
      \href{https://github.com/jdamcd/musicbrainz-android/commits?author=ming13}{MusicBrainz for Android},
      but I have done major improvements as well:
      \href{https://github.com/karlseguin/the-little-mongodb-book/pull/16}{The Little MongoDB PDF building},
      \href{https://github.com/keyboardsurfer/Crouton/pulls/ming13?state=closed}{Crouton’s animations and Maven support}.


  \section*{Personal}

    \subsection*{Brief}

      Hard-working and explorative. My interests are:
      mobile application development, UI design, UNIX-like
      operating systems.

    \subsection*{In detail}

      Android development is my main specialization. I prefer designing
      applications from scratch on my own following
      \href{https://developer.android.com/design/}{Android design guidelines}.
      Except guidelines I follow Jef Raskin
      and Alan Cooper ideas. For development I use open-source libraries
      stack based on Maven or Gradle build system (look at
      \href{https://github.com/ming13/gambit}{Gambit’s source code} for example). \\

      I am a big fan of Linux and I use it for many years as
      an only operating system, so any work with UNIX-like operating systems
      would be a great pleasure and productivity boost for me.
      I like automatization everywhere I could do it and I try to optimize
      work process and software as well. \\

      In my free time I enjoy learning new ideas, projects and technologies
      that correlate with my interests and I am ready to work with them. \\

      If you have any questions to me~--- I am ready to answer.

  \vfill

\end{document}
